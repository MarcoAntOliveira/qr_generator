\documentclass[letterpaper]{article}
\usepackage[legalpaper, left=1 cm, right=1cm, top=0.5cm, bottom=0.5cm] {geometry}
\date{} % Remove a exibição da data
\usepackage{xcolor}
\usepackage{listings}
\usepackage{graphicx}
\usepackage{hyperref} % Para criar links
\usepackage[utf8]{inputenc}
\usepackage[T1]{fontenc}

\definecolor{codegreen}{rgb}{0,0.6,0}
\definecolor{codegray}{rgb}{0.5,0.5,0.5}
\definecolor{codepurple}{rgb}{0.58,0,0.82}
\definecolor{backcolour}{rgb}{0.95,0.95,0.92}

\lstdefinestyle{mystyle}{
    language=HTML, % Use 'HTML' para a linguagem HTML
    basicstyle=\ttfamily\small,
    keywordstyle=\color{blue},
    stringstyle=\color{codepurple},
    commentstyle=\color{blue}\itshape,
    numbers=left,
    numberstyle=\tiny\color{orange},
    breaklines=true,
    showstringspaces=false,
}

\lstdefinestyle{pythonStyle}{
    language=Python,
    basicstyle=\ttfamily\small,
    keywordstyle=\color{blue},
    stringstyle=\color{orange},
    commentstyle=\color{orange}\itshape,
    numbers=left,
    numberstyle=\tiny\color{blue},
    breaklines=true,
    showstringspaces=false,
}

\lstdefinestyle{pythonStyle-on}{
    language=Python,
    basicstyle=\ttfamily\small\color{green},
    keywordstyle=\color{green},
    stringstyle=\color{green},
    commentstyle=\color{green}\itshape,
    numbers=left,
    numberstyle=\tiny\color{green},
    breaklines=true,
    showstringspaces=false,
}

\lstdefinestyle{pythonStyle-off}{
    language=Python,
    basicstyle=\ttfamily\small\color{red},
    keywordstyle=\color{red},
    stringstyle=\color{red},
    commentstyle=\color{red}\itshape,
    numbers=left,
    numberstyle=\tiny\color{red},
    breaklines=true,
    showstringspaces=false,
}


% Define o estilo para HTML
\lstdefinestyle{htmlStyle}{
    language=python,
    basicstyle=\ttfamily\small,
    keywordstyle=\color{blue},
    stringstyle=\color{orange},
    commentstyle=\color{blue}\itshape,
    numbers=left,
    numberstyle=\tiny\color{orange},
    breaklines=true,
    showstringspaces=false,
}

\lstdefinestyle{htmlStyle-on}{
    language=python,
    basicstyle=\ttfamily\small\color{green},
    keywordstyle=\color{green},
    stringstyle=\color{green},
    commentstyle=\color{green}\itshape,
    numbers=left,
    numberstyle=\tiny\color{green},
    breaklines=true,
    showstringspaces=false,
}

\lstdefinestyle{htmlStyle-off}{
    language=python,
    basicstyle=\ttfamily\small\color{red},
    keywordstyle=\color{red},
    stringstyle=\color{red},
    commentstyle=\color{red}\itshape,
    numbers=left,
    numberstyle=\tiny\color{red},
    breaklines=true,
    showstringspaces=false,
}


\lstdefinestyle{cssStyle}{
    language=HTML,
    basicstyle=\ttfamily\small,
    keywordstyle=\color{blue},
    stringstyle=\color{green},
    commentstyle=\color{blue}\itshape,
    numbers=left,
    numberstyle=\tiny\color{green},
    breaklines=true,
    showstringspaces=false,
}

\lstdefinestyle{cssStyle-off}{
    language=HTML,
    basicstyle=\ttfamily\small\color{red},
    keywordstyle=\color{red},
    stringstyle=\color{red},
    commentstyle=\color{red}\itshape,
    numbers=left,
    numberstyle=\tiny\color{red},
    breaklines=true,
    showstringspaces=false,
}

\lstdefinestyle{cssStyle-on}{
    language=HTML,
    basicstyle=\ttfamily\small\color{green},
    keywordstyle=\color{green},
    stringstyle=\color{green},
    commentstyle=\color{green}\itshape,
    numbers=left,
    numberstyle=\tiny\color{green},
    breaklines=true,
    showstringspaces=false,
}

\lstset{style=mystyle}
\title{\textbf{qr_code/qr_code.tex}}
\begin{document}
\maketitle

\section{interface do usuário}

\begin{lstlisting}[style=pythonStyle, title=demo.py ] 
from tkinter import *
from tkinter import font
import pyqrcode
from PIL import ImageTk, Image

root = Tk()

canvas = Canvas(root, width=400, height=600)
canvas.pack()

app_Label = Label(root, text="QR Code Generator", fg="blue", font= ("Arial", 30))
canvas.create_window(200, 50, window=app_Label)

name_label = Label(root, text= "Link name")
link_label = Label(root, text= "Link")

canvas.create_window(200, 100, window=name_label)
canvas.create_window(200, 160, window=link_label)

name_entry =Entry(root)
link_entry = Entry(root)

canvas.create_window(200, 130, window=name_entry)
canvas.create_window(200, 180, window=link_entry)


button = Button(text="Generate Qr code", command=generate)
canvas.create_window(200, 230, window=button)



root.mainloop()
\end{lstlisting}

\section{}
\begin{lstlisting}[style=pythonStyle-on,  title= função para criação do do qr code em si] 
    
def generate():
    link_name =name_entry.get()
    link = link_entry.get()
    file_name = link_name + ".png"
    url = pyqrcode.create(link)
    url.png(file_name, scale=8)
    image = ImageTk.PhotoImage(Image.open(file_name))
    image_label =Label(image = image)
    image_label.image = image
    canvas.create_window(200, 450, window=image_label)


\end{lstlisting}

\begin{lstlisting}[style=pythonStyle-on, title = linha 173 ] 
button = Button(text="Generate Qr code", command=generate)
canvas.create_window(200, 230, window=button)
\end{lstlisting}

\begin{lstlisting}[style=pythonStyle-o, ff title = linha 173 ] 
button = Button(text="Generate Qr code")
canvas.create_window(200, 230, window=button)
\end{lstlisting}


\end{document}
\section{}
\begin{lstlisting}[style=mystyle, title= ] 
    
\end{lstlisting}
\lstinputlisting[title=aula03.html]{aulas/aula03.html}

